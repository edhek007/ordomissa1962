\documentclass[12pt,a5paper,twoside]{article}
    \title{\textbf{Ordo Miss\ae\\ Forma Extraordinaria}}
    \author{}
    \date{}
    
    \usepackage[left=2.5cm, right=2cm, top=2cm, bottom=2cm]{geometry}
    \usepackage[autocompile]{gregoriotex}
    \usepackage{color}
    \usepackage{liturgialternate}
    \usepackage{bbding}
\begin{document}

\maketitle
\thispagestyle{empty}
\section{Misa Katekumen}
\instruct{Imam berlutut di kaki altar bersama petugas dan memulai Ekaristi. Ketika imam membuat tanda salib, diikuti oleh petugas.\\
S. Imam\\
M. Petugas (Server) atau umat.}
\priest{In N\'omine Patris, ~\CrossMaltese{} et F\'ilii, et Sp\'iritus Sancti. Amen}
\instruct{Imam menggabungkan kedua tangannya dan mengucapkan antifon dan Mazmur 42 bergantian dengan server.}
\priest{Intro\'ibo ad alt\'are Dei.}
\server{Ad Deum qui l\ae{}t\'if\'icat juvent\'utem meam.}
\psalmheading{Judica me - Mazmur 42}
\leslettrine{I}\'udica me, Deus, et disc\'erne causam meam de gente non sancta; ab h\'omine in\'iquo et dol\'oso \'erue me.\\
\server{Quia tu es, Deus, fortit\'udo mea; quare me reppul\'isti, et quare tristis inc\'edo, dum affl\'igit me inimicus?}
\priest{Emitte lucem tuam et verit\'atem tuam; ipsa me dedux\'erunt, et addux\'erunt in montem sanctum tuum, et in tabern\'acula tua.}
\server{Et intro\'ibo ad alt\'are Dei, ad Deum qui l\ae{}t\'ificat juvent\'utem meam.}
\priest{Confit\'ebor tibi in c\'ithara, Deus, Deus meus. Quare tristis es, \'anima mea, et quare cont\'urbas me?}
\server{Spera in Deo, qu\'oniam adhuc confit\'ebor illi, salut\'are vultus mei et Deus meus.}
\priest{Gl\'oria Patri, et F\'ilio, et Spir\'itui Sancto.}
\server{Sicut erat in princ\'ipio et nunc et semper: et in s\ae{}cula s\ae{}cul\'orum. Amen.}
\instruct{Antifon kembali diucapkan.}
\priest{Intro\'ibo ad alt\'are Dei.}
\server{Ad Deum qui l\ae{}t\'if\'icat juvent\'utem meam.}
\instruct{Imam membuat tanda salib dan sambil berkata:}
\priest{Adjut\'orium nostrum~\CrossMaltese{} in n\'omine D\'omini.}
\server{Qui fecit c\ae{}lum et terram.}

\instruct{Imam menggabungkan kedua tangan dan dengan rendah hati membungkuk imam mengucapkan Conf\'iteor. Bagian yang dicetak tebal merupakan bagian yang berbeda antara iman dengan petugas.}
\leslettrine{C}onf\'iteor Deo omnipot\'enti, be\'at\ae{} Mar\'i\ae{} semper V\'irgini, be\'ato Mich\'\ae{}li Arch\'angelo, be\'ato Joanni Babt\'ist\ae{}, sanctis Ap\'ostolis Petro et Paulo, \'omnibus Sanctis, \textbf{et vobis, fraters}: quia pecc\'avi nimis cogitati\'one, verbo et \'opere:
\instruct{Imam memukul dada 3 kali sambil mengatakan:} mea culpa, mea culpa, mea m\'axima culpa. Ideo precor be\'atam Mar\'iam semper V\'irginem, be\'atum Mich\'\ae{}lem Arch\'angelum, be\'atum Joannem Babt\'istam, sanctos Ap\'ostolos Petrum et Paulum, \'omnes Sanctos, \textbf{et vos, fraters}: or\'are pro me ad D\'ominum Deum nostrum.

\server{Misere\'atur tui omn\'ipotens Deus, et dim\'issis pecc\'atis tuis, perd\'ucat te ad vitam \ae{}t\'ernam.}
\priest{Amen.}

\instruct{Setelah imam mengucapkan Conf\'iteor, petugas mengucapkannya juga dengan perbedaan berikut.}
\leslettrine{C}onf\'iteor Deo omnipot\'enti, be\'at\ae{} Mar\'i\ae{} semper V\'irgini, be\'ato Mich\'\ae{}li Arch\'angelo, be\'ato Joanni Babt\'ist\ae{}, sanctis Ap\'ostolis Petro et Paulo, \'omnibus Sanctis, \textbf{et tibi, Pater}: quia pecc\'avi nimis cogitati\'one, verbo et \'opere:
\instruct{Petugas memukul dada 3 kali sambil mengatakan:} mea culpa, mea culpa, mea m\'axima culpa. Ideo precor be\'atam Mar\'iam semper V\'irginem, be\'atum Mich\'\ae{}lem Arch\'angelum, be\'atum Joannem Babt\'istam, sanctos Ap\'ostolos Petrum et Paulum, \'omnes Sanctos, \textbf{et te, Pater}: or\'are pro me ad D\'ominum Deum nostrum.

\instruct{Imam menggabungkan kedua tangannya dan kemudian memberikan absolusi berkata:}
\priest{Misere\'atur vestri omn\'ipotens Deus, et dim\'issis pecc\'atis tuis, perd\'ucat vos ad vitam \ae{}t\'ernam.}
\server{Amen.}
\priest{Indulg\'entiam,~\CrossMaltese{} absoluti\'onem, et remissi\'onem peccat\'orum nostr\'orum, tr\'ibuat nobis omn\'ipotens et mis\'ericors D\'ominus.}
\server{Amen.}
\instruct{Imam menundukkan kepala dan melanjutkan.}
\priest{Deus, tu conv\'ersus vivific\'abis nos.}
\server{Et plebs tua l\ae{}t\'abitur in te.}
\priest{Ost\'ende nobis, D\'omine, miseric\'ordiam tuam.}
\server{Et salut\'are tuum da nobis.}
\priest{D\'omine, ex\'audi orati\'onem meam.}
\server{Et clamor meus ad te v\'eniat.}
\priest{D\'ominus vob\'iscum.}
\server{Et cum sp\'iritu tuo.}
\instruct{Imam merentangkan tangan dan sambil mengatupkan tangannya berkata:}
\priest{Or\'emus.}
\instruct{Imam sambil meniti tangga panti imam, dalam hati mengucapkan}
\leslettrine{A}ufer a nobis, qu\ae{}sumus, D\'omine, iniquit\'ates nostras: ut ad Sancta sanct\'orum puris mere\'amur m\'entibus intro\'ire. Per Christum D\'ominum nostrum. Amen.

\instruct{Imam mencium tengah altar di tempat relikui orang kudus tersimpan, sambil berkata:}
\leslettrine{O}r\'amus te, D\'omine, per m\'erita Sanct\'orum tu\'orum, quorum rel\'iqui\ae{} hic sunt, et \'omnium Sanct\'orum: ut indulg\'ere dign\'eris \'omnia pecc\'ata mea. Amen.

\instruct{Pada high mass, imam melanjutkan mendupai altar.}
\instruct{Imam pindah ke sisi epistel (disebelah kanan dari Tabernakel) imam membuat tanda salib dan membacakan introit}
\psalmheading{\textbf{INTROIT}}

\instruct{Setelah selesai membaca introit, imam kembali ketengah altar menggabungkan tangannya sambil berkata-kata bergantian dengan server.}

\priest{K\'yrie, el\'eison.}
\server{K\'yrie, el\'eison.}
\priest{K\'yrie, el\'eison.}
\server{Christe, el\'eison.}
\priest{Christe, el\'eison.}
\server{Christe, el\'eison.}
\priest{K\'yrie, el\'eison.}
\server{K\'yrie, el\'eison.}
\priest{K\'yrie, el\'eison.}

\instruct{Imam mengawali Gloria dengan menyanyikan bagian awal ini, yang kemudian dilanjutkan lagu gregorian secara keseluruhan:}
\psalmheading{Masa Paskah}
\gregorioscore{priest_ky--gloria_i--solesmes.1}
\psalmheading{Pesta I kelas I}
\gregorioscore{priest_ky--gloria_ii--solesmes.1}
\psalmheading{Pesta II kelas I}
\gregorioscore{priest_ky--gloria_iv--solesmes.1}
\psalmheading{Pesta Santa Perawan Maria I}
\gregorioscore{priest_ky--gloria_ix--solesmes.1}
\psalmheading{Hari Minggu sepanjang tahun}
\gregorioscore{priest_ky--gloria_xi--solesmes.1}
\psalmheading{Pesta III kelas I}
\gregorioscore{priest_ky--gloria_xii--solesmes.1}
\psalmheading{Masa Natal}
\gregorioscore{priest_ky--gloria_xv--solesmes.1}
\instruct{Selanjutnya dinyanyikan Gloria. Selama Gloria dinyayikan, semua umat berdiri. Ketika Gloria diucapkan, imam berdiri di tengah-tengah altar, merentangkan lalu mengatupkan kembali kedua tangannya, membungkuk sedikit dan berkata,}
\priest{Et in terra pax hom\'inibus bon\ae{} volunt\'atis. Laud\'amus te. Bened\'icimus te. Ador\'amus te. Glorific\'amus te. Gr\'atias \'agimus tibi propter magnam gl\'oriam tuam. D\'omine Deus, Rex c\ae{}l\'estis, Deus Pater omn\'ipotens. D\'omine Fili unig\'enite, Iesu Christe. D\'omine Deus, Agnus Dei, F\'ilius Patris. Qui tollis pecc\'ata mundi, miser\'ere nobis. Qui tollis pecc\'ata mundi, s\'uscipe deprecati\'onem nostram. Qui sedes ad d\'exteram Patris, miser\'ere nobis. Qu\'oniam tu solus Sanctus. Tu solus D\'ominus. Tu solus Alt\'issimus, Iesu Christe. Cum Sancto Sp\'iritu, in gl\'oria Dei Patris. Amen.}

\instruct{Gloria dihilangkan pada masa Puasa, Advent dan misa arwah.}
\instruct{Setelah paduan suara selesai menyanyikan Gloria, imam mencium altar, memandang ke arah umat dan mengatakan:}
\priest{Dominus vobiscum.}
\server{Et cum spiritu tuo.}
\instruct{Imam kembali ke rubrik Misa dan berkata:}
\priest{Oremus}
\instruct{Setelah berkata "Oremus", imam mengucapkan doa kolekta dengan keras dan jelas. Alangkah lebih baik, apabila dilagukan.}
\psalmheading{\textbf{Doa Kolekta}}
\instruct{Setelah imam selesai membaca atau melagukan doa kolekta, umat menjawab:}
\server{Amen.}
\psalmheading{\textbf{Epistola}}
\instruct{Selama Epistola dibacakan, umat duduk. Respon setelah pembacaan Epistula sebagai berikut:}
\server{Deo Gr\'atias}
\instruct{Setelah pembacaan Epistula, dilanjutkan dengan Gradual atau Tractus dan Alleluia. Gradual atau Tractus dan Alleluia dinyayikan oleh paduan suara, sedangkan imam membaca sendiri Gradual atau Tractus dan Alleluia. Ketika Alleluia dinyanyikan oleh paduan suara, umat berdiri.}
\psalmheading{\textbf{Gradual/Tractus dan Alleluia}}
\instruct{Petugas memindahkan rubrik misa ke bagian kitab suci (dipindahkan ke bagian kiri Tabernakel) sementara imam membungkuk di tengah-tengah altar dengan kedua tangan bersatu dan berkata:}
\leslettrine{M}unda cor meum ac l\'abia mea, omn\'ipotens Deus, qui l\'abia Isa\'i\ae{} proph\'et\ae{} c\'alculo mund\'asti ign\'ito: ita me tua grata miserati\'one dign\'are mund\'are, ut sanctum Evang\'elium tuum digne v\'aleam nunti\'are. Per Christum Dominum nostrum. Amen.\\
Iube, D\'omine, bened\'icere.

\instruct{Imam melanjutkan:}
\leslettrine{D}\'ominus sit in corde tuo et in l\'abiis tuis: ut digne et compet\'enter ann\'unties Evang\'elium suum: In n\'omine Patris, et F\'ilii, ~\CrossMaltese{} et Sp\'iritus Sancti. Amen.

\psalmheading{\textbf{Bacaan Injil}}

\instruct{Imam beralih ke arah buku Kitab Suci dari sisi altar. Imam mengatakan:}
\priest{D\'ominus vob\'iscum.}
\server{Et cum sp\'iritu tuo.}
\priest{~\CrossMaltese{} Sequ\'entia sanctii Evangelii sec\'undum Matth\ae{}um (/Marcum/Lucam/Ioannem).}
\instruct{Apabila bacaan Injil merupakan awal dari Injil tersebut, maka imam mengatakan ini:}
\priest{~\CrossMaltese{} Initium sanctii Evangelii sec\'undum Matth\ae{}um (/Marcum/Lucam/Ioannem).}
\server{Gl\'oria tibi, D\'omine.}
\instruct{Injil dibacakan. Setelah Injil selesai dibacakan, imam memberi tanda dan umat menjawab:}
\server{Laus tibi, Christe.}
\instruct{Imam mencium Kitab Suci dan berkata.}
\priest{Per evang\'elica dicta, dele\'antur nostra del\'icta}

\psalmheading{\textbf{Homili}}

\instruct{Setelah selesai homili, imam kembali ke tengah altar dan mulai menyanyikan Credo dan dilanjutkan paduan suara. Umat berdiri.}
\psalmheading{Credo I, II, IV}
\gregorioscore{priest_credo_i}
\psalmheading{Credo III}
\gregorioscore{priest_credo_iii}
\instruct{Imam melanjutkan membaca credo, sedangkan umat bersama paduan suara menyanyikannya.}
Patrem omnipot\'entem, fact\'orem c\ae{}li et terr\ae{}, visib\'ilium \'omnium et invisib\'ilium. Et in unum D\'ominum Iesum Christum, F\'ilium Dei unig\'enitum. Et ex Patre natum ante \'omnia s\'\ae{}cula. Deum de Deo, lumen de l\'umine, Deum verum de Deo vero. G\'enitum, non factum consubstanti\'alem Patri: per quem \'omnia facta sunt. Qui propter nos h\'omines et propter nostram sal\'utem desc\'endit de c\ae{}lis.

\instruct{Pada bagian ini, semua berlutut.}
et incarn\'atus est de Sp\'iritu Sancto ex Mar\'ia V\'irgine: et homo factus est. \instruct{bangkit/kembali berdiri.}
Crucif\'ixus \'etiam pro nobis: sub P\'ontio Pil\'ato passus, et sep\'ultus est. Et resurr\'exit t\'ertia die, sec\'undum Script\'uras. Et asc\'endit in c\ae{}lum: sedet ad d\'exteram Patris. Et \'iterum vent\'urus est cum gl\'oria iudic\'are vivos et m\'ortuos: cuius regni non erit finis. Et in Sp\'iritum Sanctum, D\'ominum, et vivific\'antem: qui ex Patre Fili\'oque proc\'edit. Qui cum Patre et F\'ilio simul ador\'atur et conglorific\'atur: qui loc\'utus est per Proph\'etas. Et unam sanctam cath\'olicam et apost\'olicam Eccl\'esiam. Confiteor umum bapt\'isma in remissi\'onem peccat\'orum. Et exsp\'ecto resurrecti\'onem mortu\'orum. Et vitam ~\CrossMaltese{} vent\'uri s\'\ae{}culi. Amen.

\section{Misa Kaum Beriman}
\instruct{Imam mencium altar dan berpaling ke arah umat mengatakan:}
\priest{D\'ominus vob\'iscum.}
\server{Et cum sp\'iritu tuo.}
\priest{Or\'emus.}
\psalmheading{Offertorium}
\instruct{Pada tengah altar, imam membacakan ayat offertorium. Setelah imam selesai membacakan ayat offertorium, petugas membunyikan lonceng 1 kali. Pada saat lonceng dibunyikan satu kali, umat duduk. Umat selanjutnya berdiri pada saat pr\ae{}fation hingga sanctus. Paduan suara mulai menyanyikan lagu persembahan.}
\instruct{Imam mengambil patena dengan hosti, mempersembahkan hosti dan mengatakan:}
\leslettrine{S}\'uscipe, sancte Pater, omn\'ipotens \ae{}t\'erne Deus, hanc immacul\'atam h\'ostiam, quam ego ind\'ignus f\'amulus tuus \'offero tibi Deo meo vivo et vero, pro innumerab\'ilibus pecc\'atis, et offensi\'onibus, et negleg\'entiis meis, et pro \'omnibus circumst\'antibus, sed et pro \'omnibus fid\'elibus christi\'anis vivis atque def\'unctis: ut mihi et illis prof\'iciat ad sal\'utem in vitam \ae{}t\'ernam. Amen.

\instruct{Imam membuat tanda salib dengan patena, lalu meletakkan patena di atas korporal. Selanjutnya Imam membawa piala menuju ke sisi Epistola, lalu menuang anggur ke dalam piala. Ia menambahkan sedikit air sambil berkata}
\leslettrine{D}eus ~\CrossMaltese{}, qui hum\'an\ae{} subst\'anti\ae{} dignit\'atem mirab\'iliter condid\'isti, et mirab\'ilus reform\'asti: da nobis per hujus aqu\ae{} et vini myst\'erium, eius divinit\'atis esse cons\'ortes, qui humanit\'atis nostr\ae{} f\'ieri dign\'atus est p\'articeps, Iesus Christus, F\'ilius tuus, D\'ominus noster: Qui tecum vivit et regnat in unit\'ate Sp\'iritus Sancti Deus: per \'omnia s\'\ae{}cula s\ae{}cul\'orum. Amen.

\instruct{Imam kembali ke tengah altar membawa piala berisi anggur, lalu mengangkat piala tersebut sambil berkata}
\leslettrine{O}ff\'erimus tibi, D\'omine, c\'alicem salut\'aris, tuam deprec\'antes clem\'entiam: ut in consp\'ectu div\'in\ae{} maiest\'atis tu\ae{}, pro nostra et totius mundi sal\'ute, cum od\'ore suavit\'atis asc\'endat. Amen.

\instruct{Imam membuat tanda salib dengan piala, lalu meletakkan piala di atas korporal dan menutupnya. Kemudian imam mengatupkan tangan di atas altar, membungkuk dan berkata lirih:}
\leslettrine{I}n sp\'iritu humilit\'atis et in \'animo contr\'ito suscipi\'amur a te, D\'omine: et sic fiat sacrif\'icum nostrum in consp\'ectu tuo h\'odie, ut pl\'aceat tibi, D\'omine Deus.

\instruct{Imam berdiri tegak dan mengatupkan tangan, lalu berkata sambil menandai piala dan hosti sekaligus}
\leslettrine{V}eni, sanctific\'ator omn\'ipotens \ae{}t\'erne Deus: B\'enedict Oblata, prosequendo: et b\'ene~\CrossMaltese{}dic hoc sacrificium, tuo sancto n\'omini pr\ae{}par\'atum.

\noindent\makebox[\linewidth]{\rule{\paperwidth}{0.4pt}}
\psalmheading{Khusus \emph{high mass}: \textbf{Pendupaan Persembahan}}
\instruct{Imam mengisi dupa ke \emph{thurible}/wiruk yang disiapkan oleh petugas sambil berkata:}
\leslettrine{P}er intercessi\'onem be\'ati Mich\'\ae{}lis Arch\'angeli, stantis a dextris alt\'aris inc\'ensi, et \'omnium elect\'orum su\'orum, inc\'ensum istud dign\'etur D\'ominus b\'ene~\CrossMaltese{}dicere, et in od\'orem suavit\'atis acc\'ipere. Per Christum, D\'ominum nostrum. Amen.

\instruct{Imam menerima \emph{thurible}/wiruk, lalu mendupai persembahan sambil berkata:}
\leslettrine{I}nc\'ensum istud a te bened\'ictum asc\'endat ad te, D\'omine: et desc\'endat super nos miseric\'ordia tua.

\instruct{Sambil mendaraskan mazmur berikut, imam mendupai salib dan altar. Setiap kali melintas di depan tabernakel, imam dan pelayan berlutut.}
\psalmheading{Mazmur 140: 2-4}
\leslettrine{D}irig\'atur, D\'omine, or\'atio mea, sicut inc\'ensum, in consp\'ectu tuo: elev\'atio m\'anuum me\'arum sacrif\'icium vespert\'inum. Pone, D\'omine, cust\'odiam ori meo, et \'ostium circumst\'anti\ae{} l\'abiis meis: ut non decl\'inet cor meum in verba mal\'iti\ae{}, ad excus\'andas excusati\'ones in pecc\'atis.

\instruct{Imam mengembalikan \emph{thurible}/wiruk kepada petugas dan berkata}
\leslettrine{A}cc\'endat in nobis D\'ominus ignem sui am\'oris, et flammam \ae{}t\'ern\ae{} carit\'atis. Amen.

\instruct{Setelah itu, petugas mendupai imam.}
\noindent\makebox[\linewidth]{\rule{\paperwidth}{0.4pt}}

\instruct{Di sisi Epistola, imam membasuh tangannya sambil mendaraskan doa berikut:}
\psalmheading{Mazmur 25: 6-12}
\leslettrine{L}av\'abo inter innoc\'entes manus meas: et circ\'umdabo alt\'are tuum. D\'omine: Ut \'audiam vocem laudis, et en\'arrem univ\'ersa mirab\'ilia tua. D\'omine, dil\'exit dec\'orem domus tu\ae{} et locum habitati\'onis gl\'ori\ae{} tu\ae{}. Ne perdas cum \'impiis, Deus, \'animam meam, et cum viris s\'anguinum vitam meam: In quorum m\'anibus iniquit\'ates sunt: d\'extera e\'orum repl\'eta est mun\'eribus. Ego autem in innoc\'entia mea ingr\'essus sum: r\'edime me et miser\'ere mei. Pes meus stetit in dir\'ecto: in eccl\'esiis bened\'icam te, D\'omine.

\instruct{Setelah selesai mendaraskan mazmur tersebut, dilanjutkan dengan kemuliaan. Kemuliaan \textbf{tidak diucapkan} ketika \textbf{misa arwah atau minggu sengsara}.}
Gl\'oria patri, et Filio, et Spir\'itui Sancto. Sicut erat in princ\'ipio, et nunc, et semper: et in s\'\ae{}cula s\ae{}cul\'orum. Amen.

\instruct{Imam kembali ke tengah altar, lalu membungkuk dan mengatupkan tangan di atas altar, lalu berkata:}
\leslettrine{S}\'uscipe, sancta Trinitas, hanc oblati\'onem, quam tibi off\'erimus ob mem\'oriam passi\'onis, resurrecti\'onis, et ascensi\'onis Iesu Christi, D\'omini nostri: et in hon\'orem be\'at\ae{} Mar\'i\ae{} semper V\'irginis, et be\'ati Ioannis Babtist\ae{}, et Sanct\'orum Apostol\'orum Petri et Pauli, et ist\'orum et \'omnium Sanct\'orum: ut illis prof\'iciat ad hon\'orem, nobis autem ad sal\'utem: et illi pro nobis interc\'edere dign\'entur in c\oe{}lis, quorum mem\'oriam \'agimus in terris. Per e\'udem Christum, D\'ominum nostrum. Amen.

\instruct{Imam mencium altar, kemudian berbalik menghadap umat dan mengucapkan dua kata ini dengan suara keras.}
\priest{Orate, fratres}

\instruct{Imam kembali menghadap altar, lalu melanjutkan:}
\priest{ut meum ac vestrum sacrif\'icium accept\'abile fiat apud Deum Patrem omnipot\'entem.}
\server{Susc\'ipiat D\'ominus sacrif\'icium de m\'anibus tuis ad laudem et gl\'oriam nominis sui, ad utilit\'atem quoque nostram toti\'usque Eccl\'esi\ae{} su\ae{} sanct\ae{}.}
\priest{Amen.}
\psalmheading{SECRETA}
\instruct{Dengan tangan terentang, imam berdoa dalam diam.}

\instruct{Imam mengakhiri secreta dengan suara keras, kemudian melagukan prefasi.}
\priestchant{8cm}{\gregorioscore{peromnia}}
%\noindent\textcolor{red}{\priestword}  
%\begin{minipage}{8cm}
%	\gregorioscore{peromnia}
%\end{minipage}
\\[1em]
\serverchant{3cm}{\gregorioscore{amin}}
%\noindent\textcolor{red}{\serverword}  
%\begin{minipage}{3cm}
%	\gregorioscore{amin}
%\end{minipage}

\psalmheading{PREFASI}
\instruct{Imam meletakkan kedua tangannya di atas altar (telapak tangan menyentuh altar). Imam bernyanyi dengan keras:}
\priestchant{6cm}{\gregorioscore{dominusvobiscum}}
\\[1em]
\serverchant{7cm}{\gregorioscore{etcumspiritutuo}}
\\[2em]
\instruct{Imam mengangkat kedua tangannya, lalu melanjutkan:}
\priestchant{6cm}{\gregorioscore{sursumcorda}}
\\[1em]
\serverchant{8cm}{\gregorioscore{habemusAdDominum}}
\\[2em]
\instruct{Imam mengatupkan tangan lalu menundukkan kepala dan melanjutkan:}
\priestchant{10cm}{\gregorioscore{gratiasAgamus}}
\\[1em]
\serverchant{8cm}{\gregorioscore{dignumetIustumest}}
\\[2em]
\instruct{Imam merentangkan tangan, kemudian menyanyikan prefasi dengan keras.}

\instruct{Pada akhir prefasi, imam mengatupkan tangan.}

\instruct{Paduan suara menyanyikan Sanctus, sedangkan imam mendaraskan sambil membungkukkan badan. Lonceng dibunyikan 2 kali pada setiap kata "Sanctus".}
\leslettrine{S}\'anctus, S\'anctus, S\'anctus D\'ominus D\'eus S\'abaoth. Pl\'eni sunt c\'\ae{}li et t\'erra gl\'oria t\'ua. Hos\'anna in exc\'elsis.
\instruct{Pada saat mengucapkan "Benedictus ...", imam berdiri tegak kembali dan membuat tanda salib di dahi dan dada.}
Bened\'ictus qui v\'enit in n\'omine D\'omini. Hos\'anna in exc\'elsis.

\psalmheading{\textbf{CANON MISS\AE}}
\instruct{Paduan suara masih bernyanyi; imam menengadah, merentangkan tangan, lalu mengatupkannya kembali, sambil berdoa dengan suara lirih.}
\leslettrine{T}e Igitur, clement\'issime Pater, per Iesum Christum, F\'ilium tuum, D\'ominum nostrum, s\'upplices rog\'amus, ac p\'etimus,

\instruct{Imam mencium altar, lalu mengatupkan tangan di depan dada.}
ut accepta habeas et bened\'icas,

\instruct{Imam tiga kali membuat tanda salib atas hosti dan piala sekaligus.}
h\ae{}c~\CrossMaltese{} dona, h\ae{}c~\CrossMaltese{} m\'unera, h\ae{}c~\CrossMaltese{} sancta sacrif\'icia illib\'ata

\instruct{Imam merentangkan tangan, melanjutkan:}
in primis, qu\ae{} tibi off\'erimus pro Eccl\'esia tua sancta cath\'olica: quam pacific\'are, custod\'ire, adun\'are et r\'egere dign\'eris toto orbe terr\'arum: una cum f\'amulo tuo Papa nostro N. ... et Ant\'istite nostro N. ... et \'omnibus orthod\'oxis, atque cath\'olic\ae{} et apost\'olic\ae{} f\'idei cult\'oribus.

\leslettrine{M}em\'ento, D\'omine, famul\'orum famular\'umque tuarum N. ... et N. ... \instruct{Nama orang yang telah meninggal atau yang diujubkan pada misa itu dapat disebutkan. Apabila tidak ada, imam mengatupkan tangan, berdoa sejenak dalam hati, lalu merentangkan tangan dan melanjutkan:}
et \'omnium circumst\'antium, quorum tibi fides c\'ognita est et nota dev\'otio, pro quibus tibi off\'erimus: vel qui tibi \'offerunt hoc sacrif\'icum laudis, pro se su\'isque \'omnibus: pro redempti\'one anim\'arum su\'arum, pro spe sal\'utis et incolumit\'atis su\ae{}: tib\'ique reddunt vota sua \ae{}t\'erno Deo vivo et vero.

\instruct{Communic\'antes dapat berbeda, sesuai dengan hari raya atau pesta gerejawi yang sedang dirayakan. Bagian communic\'antes tersebut belum tersedia pada tulisan ini.}
\leslettrine{C}ommunic\'antes, et mem\'oriam vener\'antes, in primis glori\'os\ae{} serper V\'irginis Mari\ae{}, Genitr\'icis Dei et D\'omini nostri Iesu Christi: sed et be\'ati Ioseph, ei\'usdem V\'irginis Sponsi, et beat\'orum Apostol\'orum ac M\'artyrum tu\'orum, Petri et Pauli, Andr\'e\ae{}, Iac\'obi, Io\'annis, Thom\ae{}, Iac\'obi, Phil\'ippi, Barthoolom\'\ae{}i: Lini, Cleti, Clem\'entis, Xysti, Corn\'elii, Cypri\'ani, Laur\'entii, Chrys\'ogoni, Io\'annis et Pauli, Cosm\ae{} et Dami\'ani: et \'omnium Sanct\'orum tu\'orum; quorum m\'eritis precib\'usque conc\'edas, ut in \'omnibus protecti\'onis tu\ae{} muni\'amur aux\'ilio. \instruct{Imam mengatupkan tangan} Per e\'udem Christum D\'ominum nostrum. Amen.

\instruct{Imam mengulurkan tangan di atas persembahan, lalu melanjutkan. Petugas membunyikan lonceng 1 kali.}
\leslettrine{H}anc igitur oblati\'onem servit\'utis nostr\ae{}, sed et cunct\ae{} fam\'ili\ae{} tu\ae{}, qu\'\ae{}sumus, D\'omine, ut plac\'atus acc\'ipias: di\'esque nostros in tua pace disp\'onas, atque ab \ae{}t\'erna damnati\'one nos \'eripi, et in elect\'orum tu\'orum i\'ubeas grege numer\'ari. \instruct{Imam mengatupkan tangan} Per e\'udem Christum D\'ominum nostrum. Amen.

\leslettrine{Q}uam oblati\'onem tu, Deus, in benedictam, adscr\'ip~\CrossMaltese{}tam, ra~\CrossMaltese{}tam, ration\'abilem, acceptabil\'emque f\'acere dign\'eris: ut nobis Cor~\CrossMaltese{}pus, et Sang~\CrossMaltese{}uis fiat dilect\'issimi F\'ilii tui, D\'omini nostri Iesu Christi.

\leslettrine{Q}ui pr\'idie quam pater\'etur, acc\'epit panem in sanctas ac vener\'abiles manus suas, et elev\'atis \'oculis in c\oe{}lum ad te Deum, Patrem suum omnipot\'entem, tibi gr\'atias agens, bene~\CrossMaltese{}dixit, fregit, ded\'itque disc\'ipulis suis, dicens: Acc\'ipite, et manduc\'ate ex hoc oomnes.

\instruct{Imam membungkuk, memegang hosti dengan ibu jari dan telunjuk kedua tangan, sambil berkata:}
\begin{center}
	\vspace{-1em}
	{\noindent\normalsize \textbf{Hoc est enim Corpus meum}}
\end{center}
\instruct{Imam memegang hosti, berlutut (petugas membunyikan lonceng 1 kali), berdiri, mengangkat tinggi hosti (petugas membunyikan lonceng 3 kali), lalu menaruhNya di atas korporale dan berlutut lagi. Setelah ini, ibu jari dan telunjuk imam harus tetap melekat.}

\instruct{Imam berdiri lagi, lalu melanjutkan:}
\leslettrine{S}\'imili modo postquam c\oe{}n\'atum est acc\'ipiens et hunc pr\ae{}cl\'arum C\'alicem in sanctas ac vener\'abiles manus suas tibi gr\'atias bene~\CrossMaltese{}dixit, ded\'itque disc\'ipulis suis, dicens: Acc\'ipite, et b\'ibite ex eo omnes,

\instruct{Imam membungkuk, sedikit mengangkat piala dengan kedua tangan}
\begin{center}
	\vspace{-1em}
	{\noindent\normalsize \textbf{Hic est enim Calix Sanguinis mei, novi et \ae{}t\'erni testam\'enti: myst\'erium f\'idei: qui pro vobis et pro multis effund\'etur in remissi\'onem peccat\'orum.}}
\end{center}
\instruct{Imam menaruh piala di atas korporal, lalu melanjutkan:}
H\ae{}c quotiesc\'umque fec\'eritis, in mei mem\'oriam faci\'etis.

\instruct{Imam berlutut (petugas membunyikan lonceng 1 kali), berdiri lagi, lalu mengangkat tinggi piala (petugas membunyikan lonceng 3 kali), lalu meletakkanNya kembali di atas korporal dan menutupnya. Imam berlutut lagi, berdiri, lalu merentangkan tangan dan melanjutkan doa berikut dengan suara lirih.}
\leslettrine{U}nde et m\'emores, D\'omine, nos servi tui, sed et Christi F\'ilii tui, D\'omini nostri, tam be\'at\ae{} passi\'onis, nec non et ab \'inferis ressurecti\'onis, sed et in c\oe{}los glori\'os\ae{} ascensi\'onis: off\'erimus pr\ae{}cl\'ar\ae{} maiest\'ati tu\ae{} de tui donis ac datis, h\'ostiam~\CrossMaltese{} puram, h\'ostiam~\CrossMaltese{} sanctam, h\'ostiam~\CrossMaltese{} immacul\'atam,

\instruct{Imam menandai hosti}
panem~\CrossMaltese{} sanctum vit\ae{} \ae{}t\'ern\ae{},

\instruct{Imam menandai piala}
et Calinem~\CrossMaltese{} sal\'utis perp\'etu\ae{}.

\instruct{Imam merentangkan tangan, lalu melanjutkan}
\leslettrine{S}upra qu\ae{} prop\'itio ac ser\'eno vultu respicere dign\'eris: et acc\'epta hab\'ere, sic\'uti acc\'epta hab\'ere dign\'atus es m\'unera p\'ueri tui iusti Abel, et sacrif\'icium Patri\'arch\ae{} nostri Abrah\ae{}: et quod tibi \'obtulit summus sac\'erdos tuus Melch\'isedech, sanctum sacrif\'icium, immacul\'atam h\'ostiam.

\instruct{Sambil membungkuk lalu meletakkan tangannya yang terkatup di atas altar, Imam melanjutkan.}
\leslettrine{S}\'upplices te rog\'amus, omn\'ipotens Deus: iube h\ae{}c perf\'erri per manus sancti Angeli tui in subl\'ime alt\'are tuum, in con sp\'ectu div\'in\ae{} maiest\'atis tu\ae{}: ut, quotquot ex hac alt\'aris participati\'one sacros\'anctum F\'ilii tui Cor~\CrossMaltese{}pus, et S\'an~\CrossMaltese{}guinem sumps\'erimus, omni benedicti\'one c\oe{}l\'esti et gr\'atia reple\'amur. Per e\'udem Christum, D\'ominum nostrum. Amen.

\instruct{Imam merentangkan tangan, lalu melanjutkan}
\leslettrine{M}em\'ento \'etiam, D\'omine, famul\'orum famular\'umque tu\'arum N. ... et N. ..., qui nos pr\ae{}cess\'erunt cum signo f\'idei, et d\'ormiunt in somno pacis. Ipsis, D\'omine, et \'omnibus in Christo quiesc\'entibus locum refrig\'erii, lucis pacis ut ind\'ulgeas, deprec\'amur. Per e\'udem Christum, D\'ominum nostrum. Amen.

\instruct{Imam menebah dada dan bersuara keras mengucapkan 3 kata awal.}
\leslettrine{N}obis quoque peccat\'oribus f\'amulis tuis, de multit\'udine miserati\'onum tu\'arum sper\'antibus, patrem \'aliquam et societ\'atem don\'are dign\'eris, cum tuis sanctis Ap\'ostolis et Mart\'yribus: cum Io\'anne, St\'ephano, Matth\'ia, B\'arnaba, Ign\'atio, Alex\'andro, Marcell\'ino, Petro, Felicit\'ate, Perp\'etua, Agatha, L\'ucia, Agn\'ete, C\ae{}c\'ilia, Anast\'asia, et \'omnibus Sanctis tuis: intra quorum nos cons\'ortium, non \ae{}stim\'ator m\'eriti, sed v\'eni\ae{}, qu\'\ae{}sumus, larg\'itor adm\'itte. Per Christum, D\'ominum nostrum.

\leslettrine{P}er quem h\ae{}c \'omnia, D\'omine, semper bona creas, sancti~\CrossMaltese{}ficas, vivi~\CrossMaltese{}ficas, bene~\CrossMaltese{}d\'icis et pr\ae{}stas nobis.

\instruct{Imam membuka piala, berlutut, berdiri kembali, mengambil hosti dengan ibu jari dan telunjuk kanan, memegang piala dengan tangan kiri, lalu tiga kali menandai piala dengan hosti yang dipegangnya sambil berkata lirih.}
Per ip~\CrossMaltese{}sum, et cum ip~\CrossMaltese{}so, et in ip~\CrossMaltese{}so, est tibi Deo Patri ~\CrossMaltese{} omnipotenti, in unit\'ate Sp\'iritus ~\CrossMaltese{} Sancti omnis honor, et gl\'oria,

\instruct{Imam sedikit mengangkat piala dan hosti, ketika mengucapkan "... omnis honor, gl\'oria,". Imam meletakkan hosti dan piala serta menutupnya, kemudian berlutut, lalu berdiri kembali dan berseru/bernyanyi denga keras:}
\priestchant{8cm}{\gregorioscore{peromnia}}
\\[1em]
\serverchant{3cm}{\gregorioscore{amin}}
\\[1em]
\instruct{Imam mengatupkan tangan, lalu menyanyikan ajakan berikut:}
\gregorioscore{ajakanpaternoster}

\instruct{Imam merentangkan tangan, kemudian melanjutkan dengan doa Bapa Kami. Umat \textbf{tidak perlu menyanyi} bersama imam. Silakan doakan Bapa Kami dalam hati}
\gregorioscore{pater_noster}
\instruct{Petugas dan umat menjawab pada bagian akhir:}
\serverchant{8cm}{\gregorioscore{sed_libera_nos_a_malo}}

\instruct{Imam mengatupkan tangan, lalu menjawab secara rahasia atau lirih:}
\priest{Amen.}

\instruct{Imam mengambil patena antara jari-jarinya, lalu melanjutkan dengan lirih,}
\leslettrine{L}\'ibera nos, qu\'\ae{}sumus, D\'omine, ab \'omnibus malis, pr\ae{}t\'eritis, pr\ae{}s\'entibus et fut\'uris: et interced\'ente be\'ata et glori\'osa sempert V\'irgine Dei Genetr\'ice Mar\'ia, cum be\'atis Ap\'ostolis tuis Petro et Paulo, atque Andr\'ea, et \'omnibus Sanctis,

\instruct{Imam membuat tanda salib dengan patena}
da prop\'itius pacem in di\'ebus nostris:

\instruct{Imam mencium patena}
ut, ope misericordi\ae{} tu\ae{} adi\'uti, et pecc\'ato simus semper l\'iberi et ab omni perturbati\'one sec\'uri.

\instruct{Imam meletakkan patena di atas korporal, membuka tutup piala, membungkuk, mengambil hosti dan memecahnya menjadi dua bagian di atas  piala sambil berkata.}
Per e\'undem D\'ominum nostrum Iesum Christum, F\'ilium tuum.

\instruct{setengah hosti (yang berada di tangan kanan imam) diletakkan di atas patena, setengah hosti sisanya (di tangan kiri) dipotong kecil sambil berkata:}
Qui tecum vivit et regnat in unit\'ate Sp\'iritus Sancti Deus.

\instruct{Potongan hosti yang lebih besar di tangan kiri diletakkan di atas patena, sedangkan potongan kecil masih dipegang.}
\priestchant{8cm}{\gregorioscore{peromnia}}
\\[1em]
\serverchant{3cm}{\gregorioscore{amin}}

\instruct{Imam menandai dengan partikel hosti yang dipegangnya.}
\priestchant{9cm}{\gregorioscore{paxdominisit}}
\\[1em]
\serverchant{7cm}{\gregorioscore{etcumspiritutuo2}}

\instruct{Setelah jawaban petugas dan umat tersebut, paduan suara dapat menyanyikan Agnus Dei. Imam menempatkan partikel hosti ke dalam piala, lalu berdoa dengan lirih.}
\leslettrine{H}\ae{}c commixtio, et consecr\'atio C\'orporis et S\'anguinis D\'omini nostri Iesu Christi, fiat accipi\'entibus nobis in vitam \ae{}t\'ernam. Amen.

\instruct{Imam menutup piala, berlutut, lalu berdiri kembali.}
\instruct{Sementara paduan suara menyanyikan Agnus Dei, Imam membungkuk di tengah altar sambil menebah dada dan mendaraskan Agnus Dei dengan suara lirih.}
Agnus Dei, qui tollis pecc\'ata mundi: miser\'ere nobis.\\[1em]
Agnus Dei, qui tollis pecc\'ata mundi: miser\'ere nobis.\\[1em]
Agnus Dei, qui tollis pecc\'ata mundi: dona nobis pacem.

\instruct{Imam membungkuk, meletakkan tangannya yang terkatup di atas altar sambil mendaras doa-doa berikut dengan suara lirih.}
\leslettrine{D}\'omine Iesu Christe, qui dix\'istri Ap\'ostolis tuis: Pacem rel\'inquo vobis, pacem meam do vobis: ne resp\'icias pecc\'ata mea, sed fidem Eccl\'esi\ae{} tu\ae{}; e\'amque sec\'undum volunt\'atem tuam pacific\'are et coadun\'are dign\'eris: Qui vivis et regnas Deus per \'omnia s\'\ae{}cula s\ae{}culorum. Amen.

\leslettrine{D}\'omine Iesu Christe, Fili Dei vivi, qui ex volunt\'ate Patris, cooper\'ante Sp\'iritu Sancto, per mortem tuam mundum vivific\'asti: libera me per hoc sacros\'anctum Corpus et S\'anguinem tuum ab \'omnibus iniquit\'atibus meis, et univ\'ersis malis: et fac me tuis semper inh\ae{}r\'ere mand\'atis, et a te numquam separ\'ari perm\'ittas: Qui cum e\'odem Deo Patre et Sp\'iritu Sancto vivis et regnas Deus in s\'\ae{}cula s\ae{}cul\'orum. Amen.

\leslettrine{P}erc\'eptio C\'orporis tui, D\'omine Iesu Christe, quod ego ind\'ignus s\'umere pr\ae{}s\'umo, non mihi prov\'eniat in iud\'icum et condemnati\'onem: sed pro tua plet\'ate prosit mihi ad tutam\'entum mentis et c\'orporis, et ad med\'elam percipi\'endam: Qui vivis et regnas cum Deo Patre in unit\'ate Sp\'iritus Sancti Deus, per \'omnia s\'\ae{}cula s\ae{}cul\'orum. Amen.

\instruct{Imam berlutut, berdiri kembali, lalu meletakkan tangannya di atas altar sambil mengatakan}
\priest{Panem c\oe{}l\'estem accipiam, et nomen D\'omini invoc\'abo.}

\instruct{Imam mengambil kedua potongan hosti dengan ibu jari dan telunjuk kirinya, mengambil patena dengan jari tengah dan telunjuk kirinya, sehingga ia memegang hosti dan patena bersamaan dengan tangan kirinya}

\instruct{Imam lalu membungkuk, berkata dengan lantang dan menebah dadanya (3 kali), setiap kali "Domine non sum dignus" diucapkan. Kemudian frasa lanjutannya dikatakan secara lirih.}
\priest{D\'omine non sum dignus, ut intres sub tectum meum: sed tantum dic verbo, et san\'abitur \'anima mea.}

\instruct{Imam menandai hosti dan patena dengan tangan kanannya, berkata}
\priest{Corpus D\'omini nostri Iesu Christi cust\'odiat \'animam meam in vitam \ae{}t\'ernam. Amen.}

\instruct{Sambil membungkuk, imam menyantap Tubuh Kristus, lalu tegak kembali dan hening dengan tangan terkatup.}
\instruct{Setelah itu, imam membuka piala, berlutut dan berdiri kembali, lalu memasukkan remah-remah hosti di atas patena ke dalam piala sembari berkata lirih:}
Qui retr\'ibuam D\'omino pro \'omnibus, qu\ae{} retr\'ibuit mihi? C\'alicem salut\'aris acc\'ipiam, et nomen D\'omini invoc\'abo. Laudans invoc\'abo D\'ominum, et ab inim\'icis meis salvus ero.

\instruct{Imam memegang piala dengan jari tengah dan telunjuk kanannya, lalu membuat tanda salib ke arah tabernakel dengan piala yang dipegangnya sembari berkata dengan lirih:}
Sanguis D\'omini nostri Iesu Christi cust\'odiat \'animam meam in vitam \ae{}t\'ernam. Amen.

\instruct{Kemudian, imam meminum Darah Kristus sambil memegang patena di bawah dagu dengan jari tengah dan telunjuk kiri, lalu menaruh kembali patena dan piala di atas korporale, mengelap bibir piala dengan purificatorium dan penutup piala.}
\instruct{Kemudian imam berlutut, membuka sibori, memegang sibori dengan tangan kiri dan hosti di atas sibori dengan tangan kanan, lalu menghadap umat sambil berkata dengan lantang.}
\priest{Ecce Agnus Dei, ecce qui tollit pecc\'ata mundi.}

\instruct{Imam berkata dengan lantang (3 kali) yang diikuti umat. Ketika umat mengikuti kalimat ini, sebanyak 3 kali juga menebah dada setiap frasa "Domine non sum dignus". Petugas membunyikan lonceng ketika umat dan imam mengucapkan "Domine non sum dignus".}
D\'omine non sum dignus, ut intres sub tectum meum: sed tantum dic verbo, et san\'abitur \'anima mea.

\psalmheading{\textbf{Penerimaan Komuni}}
\instruct{Imam menuju ke rel altar tempat penerimaan komuni. Umat dan petugas yang menyambut komuni berlutut, lalu membuka mulut. Imam meletakkan Tubuh Kristus di lidah masing-masing penyambut sambil berkata}
\priest{Corpus D\'omini nostri Iesu Christi cust\'odiat \'animam meam in vitam \ae{}t\'ernam. Amen.}

\instruct{Sementara itu, paduan suara dapat menyanyikan antifon komuni dan/atau lagu lain untuk mengiringi penerimaan komuni.}

\psalmheading{\textbf{Pembersihan Bejana}}
\instruct{Setelah penerimaan Tubuh Kristus kepada umat selesai, Imam kembali ke tengah altar, meminta petugas mengisi piala dengan sedikit anggur. Imam lalu meminum anggur tersebut sambari berdoa dalam hati}
\priest{Quod ore s\'umpsimus, D\'omine, pura mente capi\'amus: et de munere tempor\'ali fiat nobis rem\'edium sempit\'ernum.}

\instruct{Imam menuju ke sisi epistola membawa piala dan purificatorum. Petugas menuang anggur dan air melalui jari telunjuk dan jempol imam di atas piala. Setelah itu, imam mengelap jarinya dengan purificatorum sambil berkata:}
\priest{Corpus tuum, D\'omine, quod sumpsi, et Sanguis, quem pot\'avi, adh\'\ae{}reat visc\'eribus meis: et pr\ae{}sta; ut in me non rem\'aneat sc\'elerum m\'acula, quem pura et sancta refec\'erunt sacram\'enta: Qui vivis et regnas in s\'\ae{}cula s\ae{}cul\'orum. Amen.}

\instruct{Kemudian, imam kembali ke tengah altar, meminum campuran air dan anggur tersebut, lalu membersihkan piala dengan purificatorum. Setelah itu, imam menyusun kembali piala seperti semula.}

\instruct{Kemudian, ia menuju sisi epistola, lalu dengan tangan terkatup mendaraskan antifon komuni.}
\psalmheading{\textbf{Post Communio}}
\instruct{Imam (berdiri di tengah altar) mencium altar, berbalik memandang umat, membuka tangan, kemudian memberi salam.}
\priest{D\'ominus vob\'iscum.}
\server{Et cum sp\'iritu tuo.}

\instruct{Imam menuju sisi epistola. Saat menyanyikan Oremus dengan suara lantang, Imam membuka tangan, lalu mengatupkan tangan sambil menunduk saat menyanyikan Per Dominum. Setelah doa setelah komuni, petugas dan/atau umat menjawab}
\server{Amen.}
\psalmheading{\textbf{Benedictio}}
\instruct{Imam (berdiri di tengah altar) mencium altar, berbalik memandang umat membuka tangan, kemudian memberi salam.}
\priest{D\'ominus vob\'iscum}
\server{Et cum spiritu tuo}

\instruct{Imam mengatupkan tangan, lalu melanjutkan:}
\priest{Ite, m\'issa est.}
\server{D\'eo gr\'atias.}

\instruct{Pada masa paskah, ditambahkan Halleluia, Halleluia. Ungkapan Ite, missa est dapat dinyanyikan dengan lagu tertentu.}

\instruct{Imam berbalik menghadap altar, membungkuk dan menempatkan tangannya yang terkatup di atas altar sembari berkata dengan lirih:}
\priest{Pl\'aceat tibi, sancta Trinitas, obs\'equium servit\'utis me\ae{}: et pr\ae{}sta; ut sacrif\'icium, quod \'oculis tu\ae{} maiest\'atis ind\'ignus \'obtuli, tibi sit accept\'abile, mih\'ique et \'omnibus, pro quibus illud \'obituli, sit, te miser\'ante, propiti\'abile. Per Christum D\'ominum nostrum. Amen.}

\instruct{Imam mencium altar, lalu berdiri tegak, menengadah sambil membuka tangan lalu mengatupkan tangan sembari berkata dengan lantang:}
\priest{Bened\'icat vos omn\'ipotens Deus,}

\instruct{Imam berbalik menghadap umat}
\priest{Pater, et F\'ilius, et Sp\'iritus Sancto.}
\server{Amen.}

\psalmheading{\textbf{Injil Terakhir}}
\instruct{Imam menuju ke sisi Injil, kemudian membacakan Injil terakhir. Dengan tangan terkatup, imam berseru:}
\priest{D\'ominus vob\'iscum.}
\server{Et cum sp\'iritu tuo.}

\instruct{Dengan ibu jari kanan, Imam menandai buku Injil atau altar, dahi, mulut dan dadanya, sembari dengan lantang berkata:}
\priest{Initium sancti Evang\'elii sec\'undum Ioanem.}
\server{Gl\'oria tibi, D\'omine.}
\leslettrine{I}n princ\'ipio erat Verbum, et Verbum erat apud Deum, et Deus erat Verbum. Hoc erat in princ\'ipio apud Deum. Omnia per ipsum facta sunt, et sine ipso factum est nihil, quod factum est; in ipso vita erat, et vita erat lux h\'ominum, et lux in t\'enebris lucet, et t\'enebrae eam non comprehend\'erunt.

Fuit homo missus a Deo, cui nomen erat Io\'annes; hic venit in testim\'onium, ut testim\'onium perhib\'eret de l\'umine, ut omnes cr\'ederent per illum. Non erat ille lux, sed ut testim\'onium perhib\'eret de l\'umine. Erat lux vera, qu\ae{} ill\'uminat omnem h\'ominem, veni\'entem in hunc mundum. In mundo erat, et mundus per ipsum factus est, et mundus eum non cogn\'ovit. In pr\'opria venit, et sui eum non recep\'erunt. Quotquot autem recep\'erunt eum, dedit eis potest\'atem f\'ilios Dei f\'ieri, his, qui credunt in n\'omine eius, qui non ex sangu\'inibus neque ex volunt\'ate carnis neque ex volunt\'ate viri, sed ex Deo nati sunt.

\instruct{di sini semua berlutut}
Et Verbum caro factum est

\instruct{Semua berdiri kembali dan imam melanjutkan}
et habitavit in nobis; et vidimus gl\'oriam eius, gl\'oriam quasi Unig\'eniti a Patre, plenum gr\'ati\ae{} et veritatis.

\server{Deo gr\'atias}

\instruct{Imam menuju kaki altar, berbalik menghadap altar, berlutut dan berdiri kembali lalu bersama petugas menuju sakristi. Paduan suara dapat mengiringi imam dan petugas menuju sakristi dengan lagu.}

\end{document}

